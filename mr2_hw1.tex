% Options for packages loaded elsewhere
\PassOptionsToPackage{unicode}{hyperref}
\PassOptionsToPackage{hyphens}{url}
%
\documentclass[
]{article}
\usepackage{amsmath,amssymb}
\usepackage{lmodern}
\usepackage{iftex}
\ifPDFTeX
  \usepackage[T1]{fontenc}
  \usepackage[utf8]{inputenc}
  \usepackage{textcomp} % provide euro and other symbols
\else % if luatex or xetex
  \usepackage{unicode-math}
  \defaultfontfeatures{Scale=MatchLowercase}
  \defaultfontfeatures[\rmfamily]{Ligatures=TeX,Scale=1}
\fi
% Use upquote if available, for straight quotes in verbatim environments
\IfFileExists{upquote.sty}{\usepackage{upquote}}{}
\IfFileExists{microtype.sty}{% use microtype if available
  \usepackage[]{microtype}
  \UseMicrotypeSet[protrusion]{basicmath} % disable protrusion for tt fonts
}{}
\makeatletter
\@ifundefined{KOMAClassName}{% if non-KOMA class
  \IfFileExists{parskip.sty}{%
    \usepackage{parskip}
  }{% else
    \setlength{\parindent}{0pt}
    \setlength{\parskip}{6pt plus 2pt minus 1pt}}
}{% if KOMA class
  \KOMAoptions{parskip=half}}
\makeatother
\usepackage{xcolor}
\IfFileExists{xurl.sty}{\usepackage{xurl}}{} % add URL line breaks if available
\IfFileExists{bookmark.sty}{\usepackage{bookmark}}{\usepackage{hyperref}}
\hypersetup{
  pdftitle={Assignment1},
  hidelinks,
  pdfcreator={LaTeX via pandoc}}
\urlstyle{same} % disable monospaced font for URLs
\usepackage[margin=1in]{geometry}
\usepackage{color}
\usepackage{fancyvrb}
\newcommand{\VerbBar}{|}
\newcommand{\VERB}{\Verb[commandchars=\\\{\}]}
\DefineVerbatimEnvironment{Highlighting}{Verbatim}{commandchars=\\\{\}}
% Add ',fontsize=\small' for more characters per line
\usepackage{framed}
\definecolor{shadecolor}{RGB}{248,248,248}
\newenvironment{Shaded}{\begin{snugshade}}{\end{snugshade}}
\newcommand{\AlertTok}[1]{\textcolor[rgb]{0.94,0.16,0.16}{#1}}
\newcommand{\AnnotationTok}[1]{\textcolor[rgb]{0.56,0.35,0.01}{\textbf{\textit{#1}}}}
\newcommand{\AttributeTok}[1]{\textcolor[rgb]{0.77,0.63,0.00}{#1}}
\newcommand{\BaseNTok}[1]{\textcolor[rgb]{0.00,0.00,0.81}{#1}}
\newcommand{\BuiltInTok}[1]{#1}
\newcommand{\CharTok}[1]{\textcolor[rgb]{0.31,0.60,0.02}{#1}}
\newcommand{\CommentTok}[1]{\textcolor[rgb]{0.56,0.35,0.01}{\textit{#1}}}
\newcommand{\CommentVarTok}[1]{\textcolor[rgb]{0.56,0.35,0.01}{\textbf{\textit{#1}}}}
\newcommand{\ConstantTok}[1]{\textcolor[rgb]{0.00,0.00,0.00}{#1}}
\newcommand{\ControlFlowTok}[1]{\textcolor[rgb]{0.13,0.29,0.53}{\textbf{#1}}}
\newcommand{\DataTypeTok}[1]{\textcolor[rgb]{0.13,0.29,0.53}{#1}}
\newcommand{\DecValTok}[1]{\textcolor[rgb]{0.00,0.00,0.81}{#1}}
\newcommand{\DocumentationTok}[1]{\textcolor[rgb]{0.56,0.35,0.01}{\textbf{\textit{#1}}}}
\newcommand{\ErrorTok}[1]{\textcolor[rgb]{0.64,0.00,0.00}{\textbf{#1}}}
\newcommand{\ExtensionTok}[1]{#1}
\newcommand{\FloatTok}[1]{\textcolor[rgb]{0.00,0.00,0.81}{#1}}
\newcommand{\FunctionTok}[1]{\textcolor[rgb]{0.00,0.00,0.00}{#1}}
\newcommand{\ImportTok}[1]{#1}
\newcommand{\InformationTok}[1]{\textcolor[rgb]{0.56,0.35,0.01}{\textbf{\textit{#1}}}}
\newcommand{\KeywordTok}[1]{\textcolor[rgb]{0.13,0.29,0.53}{\textbf{#1}}}
\newcommand{\NormalTok}[1]{#1}
\newcommand{\OperatorTok}[1]{\textcolor[rgb]{0.81,0.36,0.00}{\textbf{#1}}}
\newcommand{\OtherTok}[1]{\textcolor[rgb]{0.56,0.35,0.01}{#1}}
\newcommand{\PreprocessorTok}[1]{\textcolor[rgb]{0.56,0.35,0.01}{\textit{#1}}}
\newcommand{\RegionMarkerTok}[1]{#1}
\newcommand{\SpecialCharTok}[1]{\textcolor[rgb]{0.00,0.00,0.00}{#1}}
\newcommand{\SpecialStringTok}[1]{\textcolor[rgb]{0.31,0.60,0.02}{#1}}
\newcommand{\StringTok}[1]{\textcolor[rgb]{0.31,0.60,0.02}{#1}}
\newcommand{\VariableTok}[1]{\textcolor[rgb]{0.00,0.00,0.00}{#1}}
\newcommand{\VerbatimStringTok}[1]{\textcolor[rgb]{0.31,0.60,0.02}{#1}}
\newcommand{\WarningTok}[1]{\textcolor[rgb]{0.56,0.35,0.01}{\textbf{\textit{#1}}}}
\usepackage{graphicx}
\makeatletter
\def\maxwidth{\ifdim\Gin@nat@width>\linewidth\linewidth\else\Gin@nat@width\fi}
\def\maxheight{\ifdim\Gin@nat@height>\textheight\textheight\else\Gin@nat@height\fi}
\makeatother
% Scale images if necessary, so that they will not overflow the page
% margins by default, and it is still possible to overwrite the defaults
% using explicit options in \includegraphics[width, height, ...]{}
\setkeys{Gin}{width=\maxwidth,height=\maxheight,keepaspectratio}
% Set default figure placement to htbp
\makeatletter
\def\fps@figure{htbp}
\makeatother
\setlength{\emergencystretch}{3em} % prevent overfull lines
\providecommand{\tightlist}{%
  \setlength{\itemsep}{0pt}\setlength{\parskip}{0pt}}
\setcounter{secnumdepth}{-\maxdimen} % remove section numbering
\ifLuaTeX
  \usepackage{selnolig}  % disable illegal ligatures
\fi

\title{Assignment1}
\author{true \and true \and true}
\date{2023-02-15}

\begin{document}
\maketitle

\begin{Shaded}
\begin{Highlighting}[]
\FunctionTok{library}\NormalTok{(latexpdf)}
\FunctionTok{library}\NormalTok{(estimability)}
\end{Highlighting}
\end{Shaded}

\begin{verbatim}
## Warning: package 'estimability' was built under R version 4.2.1
\end{verbatim}

\begin{Shaded}
\begin{Highlighting}[]
\FunctionTok{library}\NormalTok{(gmodels)}
\end{Highlighting}
\end{Shaded}

\begin{verbatim}
## Warning: package 'gmodels' was built under R version 4.2.2
\end{verbatim}

\begin{Shaded}
\begin{Highlighting}[]
\FunctionTok{library}\NormalTok{(Sleuth3)}
\end{Highlighting}
\end{Shaded}

\begin{verbatim}
## Warning: package 'Sleuth3' was built under R version 4.2.2
\end{verbatim}

\begin{Shaded}
\begin{Highlighting}[]
\NormalTok{content\_grazers}\OtherTok{=}\NormalTok{Sleuth3}\SpecialCharTok{::}\NormalTok{case1301 }
\NormalTok{content\_grazers}\SpecialCharTok{$}\NormalTok{Treat}\OtherTok{=}\FunctionTok{factor}\NormalTok{(content\_grazers}\SpecialCharTok{$}\NormalTok{Treat,}
                        \AttributeTok{levels =}\FunctionTok{levels}\NormalTok{(content\_grazers}\SpecialCharTok{$}\NormalTok{Treat)[}\FunctionTok{c}\NormalTok{(}\DecValTok{1}\NormalTok{,}\DecValTok{5}\NormalTok{,}\DecValTok{6}\NormalTok{,}\DecValTok{2}\NormalTok{,}\DecValTok{3}\NormalTok{,}\DecValTok{4}\NormalTok{)])}
\FunctionTok{levels}\NormalTok{(content\_grazers}\SpecialCharTok{$}\NormalTok{Block)}
\end{Highlighting}
\end{Shaded}

\begin{verbatim}
## [1] "B1" "B2" "B3" "B4" "B5" "B6" "B7" "B8"
\end{verbatim}

\begin{Shaded}
\begin{Highlighting}[]
\FunctionTok{contrasts}\NormalTok{(content\_grazers}\SpecialCharTok{$}\NormalTok{Block)}
\end{Highlighting}
\end{Shaded}

\begin{verbatim}
##    B2 B3 B4 B5 B6 B7 B8
## B1  0  0  0  0  0  0  0
## B2  1  0  0  0  0  0  0
## B3  0  1  0  0  0  0  0
## B4  0  0  1  0  0  0  0
## B5  0  0  0  1  0  0  0
## B6  0  0  0  0  1  0  0
## B7  0  0  0  0  0  1  0
## B8  0  0  0  0  0  0  1
\end{verbatim}

\begin{Shaded}
\begin{Highlighting}[]
\FunctionTok{contrasts}\NormalTok{(content\_grazers}\SpecialCharTok{$}\NormalTok{Treat)}
\end{Highlighting}
\end{Shaded}

\begin{verbatim}
##     f fF L Lf LfF
## C   0  0 0  0   0
## f   1  0 0  0   0
## fF  0  1 0  0   0
## L   0  0 1  0   0
## Lf  0  0 0  1   0
## LfF 0  0 0  0   1
\end{verbatim}

\begin{Shaded}
\begin{Highlighting}[]
\NormalTok{a}\OtherTok{=}\DecValTok{8}
\NormalTok{b}\OtherTok{=}\DecValTok{6}
\end{Highlighting}
\end{Shaded}

\hypertarget{problem-1}{%
\subsection{Problem 1}\label{problem-1}}

We are testing whether limpets have a different effect when large fish are
present versus when large fish are not present. Therefore, we test the
following hypotheses: H0 : (βLfF − βfF ) − (βLf − βf ) = 0, HA : βfF ) −
(βLf − βf )= 0 Using the order of levels above and algebra, this can be
re-written as: H0 : β5 − β2 − β4 + β1 = 0, HA : β5 − β2 − β4 + β1 = 0,
βC = β0 = 0

\begin{Shaded}
\begin{Highlighting}[]
\NormalTok{data\_grazers}\OtherTok{=}\FunctionTok{lm}\NormalTok{(}\FunctionTok{log}\NormalTok{(Cover}\SpecialCharTok{/}\NormalTok{(}\DecValTok{100}\SpecialCharTok{{-}}\NormalTok{Cover))}\SpecialCharTok{\textasciitilde{}}\NormalTok{Block}\SpecialCharTok{+}\NormalTok{Treat, data}
                \OtherTok{=}\NormalTok{content\_grazers)}

\NormalTok{Coeff\_a}\OtherTok{=}\FunctionTok{rep}\NormalTok{(}\DecValTok{0}\NormalTok{,a}\DecValTok{{-}1}\NormalTok{) }
\NormalTok{Coeff\_b}\OtherTok{=}\FunctionTok{c}\NormalTok{(}\DecValTok{1}\NormalTok{,}\SpecialCharTok{{-}}\DecValTok{1}\NormalTok{,}\DecValTok{0}\NormalTok{,}\SpecialCharTok{{-}}\DecValTok{1}\NormalTok{,}\DecValTok{1}\NormalTok{)}
\NormalTok{C}\OtherTok{=}\FunctionTok{c}\NormalTok{(}\DecValTok{0}\NormalTok{,Coeff\_a,Coeff\_b)}
\NormalTok{est}\OtherTok{\textless{}{-}}\FunctionTok{estimable}\NormalTok{(data\_grazers,C,}\AttributeTok{conf.int=}\NormalTok{.}\DecValTok{95}\NormalTok{)}
\NormalTok{est}\SpecialCharTok{$}\NormalTok{Lower.CI}
\end{Highlighting}
\end{Shaded}

\begin{verbatim}
## [1] -0.808072
\end{verbatim}

\begin{Shaded}
\begin{Highlighting}[]
\NormalTok{est}\SpecialCharTok{$}\NormalTok{Upper.CI}
\end{Highlighting}
\end{Shaded}

\begin{verbatim}
## [1] 0.383042
\end{verbatim}

Ans: Given a p-value of 0.48 and α = 0.5, the difference in log chances
of seaweed regeneration falls between -0.81 and 0.38 according to the
95\% confidence interval, which means that we do not have enough data to
support the hypothesis that the effect of limpets on seaweed
regeneration is different when large fish are present compared to when
large fish are not present.

\hypertarget{problem-2}{%
\subsection{Problem 2}\label{problem-2}}

Here we are looking for the potential impact of tiny fish on the
regrowth of seaweed. H0: 1/2(βLf − βL) + 1/2(βf − βC) = 0, HA: 1/2(βLf −
βL) + 1/2(βf − βC) = 0 Using the order of levels above and algebra, this
can be re-written as: H0: 1/2 (β4 − β3 + β1) = 0, HA: 1 /2(β4 − β3 + β1)
= 0, βC = β0 = 0

\begin{Shaded}
\begin{Highlighting}[]
\NormalTok{data\_grazers}\OtherTok{=}\FunctionTok{lm}\NormalTok{(}\FunctionTok{log}\NormalTok{(Cover}\SpecialCharTok{/}\NormalTok{(}\DecValTok{100}\SpecialCharTok{{-}}\NormalTok{Cover))}\SpecialCharTok{\textasciitilde{}}\NormalTok{Block}\SpecialCharTok{+}\NormalTok{Treat, data}
              \OtherTok{=}\NormalTok{content\_grazers)}
\NormalTok{Coeff\_a}\OtherTok{=}\FunctionTok{rep}\NormalTok{(}\DecValTok{0}\NormalTok{,a}\DecValTok{{-}1}\NormalTok{)}
\NormalTok{Coeff\_b}\OtherTok{=}\NormalTok{.}\DecValTok{5}\SpecialCharTok{*}\FunctionTok{c}\NormalTok{(}\DecValTok{1}\NormalTok{,}\DecValTok{0}\NormalTok{,}\SpecialCharTok{{-}}\DecValTok{1}\NormalTok{,}\DecValTok{1}\NormalTok{,}\DecValTok{0}\NormalTok{)}
\NormalTok{C}\OtherTok{=}\FunctionTok{c}\NormalTok{(}\DecValTok{0}\NormalTok{,Coeff\_a,Coeff\_b)}
\NormalTok{est1}\OtherTok{=}\FunctionTok{estimable}\NormalTok{(data\_grazers,C,}\AttributeTok{conf.int =}\NormalTok{.}\DecValTok{95}\NormalTok{) }
\NormalTok{est1}\SpecialCharTok{$}\NormalTok{Lower.CI}
\end{Highlighting}
\end{Shaded}

\begin{verbatim}
## [1] -0.6910422
\end{verbatim}

\begin{Shaded}
\begin{Highlighting}[]
\NormalTok{est1}\SpecialCharTok{$}\NormalTok{Upper.CI}
\end{Highlighting}
\end{Shaded}

\begin{verbatim}
## [1] -0.09548521
\end{verbatim}

Ans: Given a p-value of 0.010 and α=0.5, there is compelling evidence
that microscopic fish have an effect on the regrowth of seaweed. With a
95\% level of confidence, the impact of microscopic fish on log odds
seaweed regeneration is predicted to be between -0.69 and -0.095.

\hypertarget{problem-3}{%
\subsection{Problem 3}\label{problem-3}}

Testing the effect of limpets on seaweed growth. H0 : 1/3(βL − βC) +
1/3(βLf − βf) + 1/3(βLfF − βfF ) = 0, HA : 1/3(βL − βC) + 1/3(βLf − βf)
+ 1/3(βLfF − βfF ) 0 Using the order of levels above and algebra, this
can be re-written as: H0 : 1/3(β3 + β4 − β1 + β5 − β2) = 0, HA : 1/3(β3+
β4− β1+ β5− β2)=0, βC = β0 = 0

\begin{Shaded}
\begin{Highlighting}[]
\NormalTok{data\_grazers}\OtherTok{=}\FunctionTok{lm}\NormalTok{(}\FunctionTok{log}\NormalTok{(Cover}\SpecialCharTok{/}\NormalTok{(}\DecValTok{100}\SpecialCharTok{{-}}\NormalTok{Cover))}\SpecialCharTok{\textasciitilde{}}\NormalTok{Block}\SpecialCharTok{+}\NormalTok{Treat, data}
              \OtherTok{=}\NormalTok{content\_grazers)}
\NormalTok{Coeff\_a}\OtherTok{=}\FunctionTok{rep}\NormalTok{(}\DecValTok{0}\NormalTok{,a}\DecValTok{{-}1}\NormalTok{)}
\NormalTok{Coeff\_b}\OtherTok{=}\NormalTok{(}\DecValTok{1}\SpecialCharTok{/}\DecValTok{3}\NormalTok{)}\SpecialCharTok{*}\FunctionTok{c}\NormalTok{(}\SpecialCharTok{{-}}\DecValTok{1}\NormalTok{,}\SpecialCharTok{{-}}\DecValTok{1}\NormalTok{,}\DecValTok{1}\NormalTok{,}\DecValTok{1}\NormalTok{,}\DecValTok{1}\NormalTok{)}
\NormalTok{C}\OtherTok{=}\FunctionTok{c}\NormalTok{(}\DecValTok{0}\NormalTok{,Coeff\_a,Coeff\_b)}
\NormalTok{est2}\OtherTok{=}\FunctionTok{estimable}\NormalTok{(data\_grazers,C,}\AttributeTok{conf.int =}\NormalTok{.}\DecValTok{95}\NormalTok{) }
\NormalTok{est2}\SpecialCharTok{$}\NormalTok{Lower.CI}
\end{Highlighting}
\end{Shaded}

\begin{verbatim}
## [1] -2.071954
\end{verbatim}

\begin{Shaded}
\begin{Highlighting}[]
\NormalTok{est2}\SpecialCharTok{$}\NormalTok{Upper.CI}
\end{Highlighting}
\end{Shaded}

\begin{verbatim}
## [1] -1.585684
\end{verbatim}

Ans: Given a p-value of 2.2e-16 and α=.05, there is very strong evidence
that limpets have an impact on seaweed regeneration. We have a 95\%
confidence level that limpets have an effect that is between -2.07 and
-1.56.

\end{document}
